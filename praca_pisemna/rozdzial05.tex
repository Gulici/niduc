\chapter{Podsumowanie i wnioski końcowe}

Celem niniejszej pracy była analiza skuteczności detekcji pojedynczych uszkodzeń przemijających w arytmetycznych układach samosprawdzających opartych na kodzie resztowym modulo~3. Badaniom poddano dwie struktury sumatorów binarnych: sumator z propagacją przeniesienia (RCA) oraz sumator z wyprzedzającym generowaniem przeniesień (CLA), współpracujące z torem kontrolnym realizującym obliczenia w arytmetyce modulo~3.

Modele badanych układów zostały zaimplementowane w języku Verilog i poddane symulacji w środowisku ModelSim. Przyjęty model uszkodzeń zakładał możliwość wystąpienia pojedynczego uszkodzenia typu \emph{stuck-at-0} lub \emph{stuck-at-1} na dowolnym sygnale wyjściowym elementarnych komponentów, takich jak sumatory, generatory przeniesień, moduły resztowe oraz komparator. Takie podejście umożliwiło systematyczne i jednoznaczne przetestowanie wszystkich potencjalnych punktów awarii w strukturze układu.

Przeprowadzone eksperymenty wykazały, że zarówno w przypadku struktury RCA, jak i CLA, nie wystąpił ani jeden przypadek niewykrytego błędu arytmetycznego (\emph{False Negative}). Oznacza to, że zastosowany mechanizm kontroli modulo~3 zapewnia pełne pokrycie pojedynczych uszkodzeń prowadzących do błędnego wyniku obliczeń. Z punktu widzenia niezawodności obliczeń jest to wynik jednoznacznie pozytywny i potwierdzający własności samosprawdzające badanych układów.

Jednocześnie zaobserwowano, że znaczna część wstrzykiwanych uszkodzeń w torze arytmetycznym nie wpływa na końcowy wynik sumowania. W przypadku sumatora RCA dokładnie połowa testowanych uszkodzeń nie powodowała zmiany wyniku, co wynika z naturalnej redundancji logicznej oraz maskowania błędów w strukturze propagacji przeniesień. W strukturze CLA odsetek uszkodzeń prowadzących do błędnego wyniku był niższy i wynosił około 38\%, co można wiązać z bardziej złożoną strukturą i większą ilością sygnałów wewnętrznych.

Uszkodzenia wstrzykiwane w elementy toru kontrolnego nie prowadziły do błędów arytmetycznych, lecz skutkowały występowaniem fałszywych alarmów (\emph{False Positive}). Zjawisko to było szczególnie widoczne w przypadku generatora reszt modulo~3 oraz sumatora modulo~3, gdzie liczba fałszywych alarmów była porównywalna z liczbą przypadków poprawnej pracy. Dla komparatora modulo~3 fałszywy alarm występował w około 25\% przypadków. Taki charakter błędów jest zgodny z oczekiwaniami i stanowi kompromis w układach samosprawdzających pomiędzy pełną detekcją błędów a nadmiarowością sygnałów alarmowych.

Podsumowując, wyniki pracy potwierdzają, że arytmetyczne układy samosprawdzające oparte na kodzie resztowym modulo~3 skutecznie wykrywają wszystkie pojedyncze uszkodzenia prowadzące do błędnych wyników obliczeń, zarówno w strukturach RCA, jak i CLA. Różnice pomiędzy badanymi sumatorami ujawniają się głównie w podatności toru arytmetycznego na uszkodzenia, natomiast zachowanie toru kontrolnego pozostaje niezależne od zastosowanej architektury sumatora.
