\chapter{Analiza wyników}

W niniejszym rozdziale przedstawiono analizę wyników symulacji odporności na pojedyncze uszkodzenia typu \emph{stuck-at} dla dwóch wariantów sumatora binarnego: RCA oraz CLA.  
W obu przypadkach analizie poddano zarówno tor arytmetyczny, jak i tor kontrolny oparty na kodzie resztowym modulo~3.

\section{Wyniki symulacji dla układu RCA}

\begin{lstlisting}[caption={Wyniki symulacji dla układu RCA~\ref{lst:tb_rca}},label={lst:rca_results}]
# === START SYMULACJI ===
# === RCA ref ===
# TP=0
# TN=256
# FP=0
# FN=0
# === RCA ===
# TP=2048
# TN=2048
# FP=0
# FN=0
# === Mod3 Adder ===
# TP=0
# TN=1024
# FP=1024
# FN=0
# === Komparator ===
# TP=0
# TN=768
# FP=256
# FN=0
# === Generator reszt mod3 ===
# TP=0
# TN=2048
# FP=2048
# FN=0
# === KONIEC SYMULACJI ===
\end{lstlisting}

Najważniejszą obserwacją dla układu RCA jest całkowity brak błędów niewykrytych (\textbf{FN = 0}) we wszystkich badanych przypadkach. Oznacza to, że każde uszkodzenie prowadzące do błędu arytmetycznego zostało poprawnie zasygnalizowane przez tor kontrolny.

W przypadku samego sumatora RCA uszkodzenia wstrzykiwane do toru arytmetycznego powodowały błąd wyniku w dokładnie 50\% przypadków. Pozostałe 50\% uszkodzeń nie miało wpływu na wynik obliczeń. Charakterystyczne jest przy tym równomierne rozłożenie przypadków \emph{TP} i \emph{TN}.

Uszkodzenia wprowadzane do elementów toru kontrolnego nie powodowały błędów wyniku, lecz skutkowały pojawieniem się fałszywych alarmów (\emph{FP}).  
Dla generatora reszt modulo~3 oraz sumatora modulo~3 liczba fałszywych alarmów była równa liczbie przypadków gdzie uszkodzenie nie powodowało błędu (\emph{FP = TN}), natomiast w przypadku komparatora fałszywy alarm występował w 25\% badanych przypadków.

\section{Wyniki symulacji dla układu CLA}

\begin{lstlisting}[caption={Wyniki symulacji dla układu CLA~\ref{lst:tb_cla}},label={lst:cla_results}]
# === START SYMULACJI ===
# === CLA ref ===
# TP=0
# TN=256
# FP=0
# FN=0
# === CLA ===
# TP=3140
# TN=5052
# FP=0
# FN=0
# === Mod3 Adder ===
# TP=0
# TN=1024
# FP=1024
# FN=0
# === Komparator ===
# TP=0
# TN=768
# FP=256
# FN=0
# === Generator reszt mod3 ===
# TP=0
# TN=2048
# FP=2048
# FN=0
# === KONIEC SYMULACJI ===
\end{lstlisting}

Podobnie jak w przypadku układu RCA, również dla sumatora CLA nie odnotowano żadnego błędu niewykrytego (\textbf{FN = 0}). Oznacza to, że zastosowany mechanizm samosprawdzania zachowuje pełną skuteczność detekcji pojedynczych uszkodzeń także dla bardziej złożonej struktury arytmetycznej.

W przeciwieństwie do RCA, w układzie CLA odsetek uszkodzeń prowadzących do błędu wyniku jest wyraźnie mniejszy i wynosi około 38\%. Większość wstrzykiwanych uszkodzeń zostaje zamaskowana przez strukturę sumatora, co można wiązać z równoległym wyznaczaniem przeniesień oraz większą liczbą ścieżek logicznych.

Zachowanie toru kontrolnego dla układu CLA jest identyczne jak w przypadku RCA. Uszkodzenia wprowadzane do generatora reszt modulo~3, sumatora modulo~3 oraz komparatora prowadzą wyłącznie do fałszywych alarmów i nie wpływają na poprawność wyniku arytmetycznego. Jest to bezpośrednią konsekwencją separacji toru obliczeniowego i kontrolnego oraz stanowi oczekiwany efekt przyjętej architektury.

