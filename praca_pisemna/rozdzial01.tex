\chapter{Wstęp}

\section{Cel i zakres pracy}
Celem niniejszej pracy jest analiza niezawodności układów arytmetycznych narażonych na występowanie błędów przemijających, na przykładzie czterobitowego sumatora wyposażonego w układ samosprawdzający oparty na arytmetyce resztowej modulo 3. W szczególności badaniu poddano skuteczność detekcji błędów poprzez symulację pojedynczych uszkodzeń logicznych typu \textit{stuck-at-0} oraz \textit{stuck-at-1}.

Praca obejmuje zarówno część teoretyczną - przedstawiającą podstawy działania kodów resztowych i ich właściwości w kontekście niezawodności układów cyfrowych - jak i część eksperymentalną, obejmującą symulację działania układu oraz analizę uzyskanych wyników.

\section{Tło i znaczenie problemu}
Wraz z miniaturyzacją technologii półprzewodnikowych rośnie podatność układów cyfrowych na zakłócenia promieniowania kosmicznego i jonizującego. Cząstki wysokoenergetyczne mogą powodować chwilowe zmiany stanów logicznych w tranzystorach, prowadząc do wystąpienia błędów przemijających (\textit{soft errors}). Tego typu zjawiska nie powodują trwałego uszkodzenia sprzętu, lecz mogą skutkować błędnymi wynikami obliczeń.

W celu ograniczenia wpływu takich błędów opracowano szereg metod detekcji i korekcji, w tym techniki oparte na kodach resztowych, które pozwalają na bieżące sprawdzanie poprawności obliczeń bez znaczącego wzrostu złożoności układu. Kody resztowe, a w szczególności kod modulo 3, umożliwiają realizację tzw. układów samosprawdzających, w których każda operacja arytmetyczna ma równoległy tor kontrolny pracujący w arytmetyce resztowej.

\newpage
\section{Założenia eksperymentu}
W eksperymencie zbudowano model układu kombinacyjnego składającego się z:
\begin{itemize}
    \item głównego sumatora czterobitowego,
    \item modułu generującego resztę z dzielenia przez 3 dla pięciobitowej liczby reprezentującej sumę wejść,
    \item sumatora kontrolnego realizującego operacje w arytmetyce modulo 3,
    \item modułów generujących resztę mod 3 dla operandów wejściowych,
    \item komparatora porównującego reszty wyniku z obu torów.
\end{itemize}

Kluczowym elementem jest możliwość symulowania pojedynczych uszkodzeń w strukturze układu spowodowanych promieniowaniem kosmicznym.
Dla każdego wprowadzonego uszkodzenia przeprowadzana jest pełna analiza działania układu, obejmująca porównanie wyniku obliczeń głównego toru arytmetycznego z wynikiem toru kontrolnego modulo 3.

Otrzymane wyniki klasyfikowane są w trzech kategoriach:
\begin{itemize}
\item \textbf{błąd wykryty} – wynik błędny został poprawnie zidentyfikowany przez komparator,
\item \textbf{błąd niewykryty (false negative)} – wynik błędny został uznany za poprawny,
\item \textbf{błąd fałszywy (false positive)} – wynik poprawny został zidentyfikowany jako błędny.
\end{itemize}

Takie podejście pozwala na ilościową ocenę skuteczności detekcji błędów przez układ samosprawdzający oraz identyfikację elementów najbardziej wrażliwych na uszkodzenia logiczne.
