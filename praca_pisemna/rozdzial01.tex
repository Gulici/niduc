\chapter{Wstęp}

\section{Cel i zakres pracy}
Celem niniejszej pracy jest analiza niezawodności układów arytmetycznych narażonych na występowanie błędów przemijających, na przykładzie czterobitowego sumatora wyposażonego w układ samosprawdzający oparty na arytmetyce resztowej modulo 3. W szczególności badaniu poddano skuteczność detekcji błędów poprzez symulację pojedynczych uszkodzeń logicznych typu \textit{stuck-at-0} oraz \textit{stuck-at-1}.

Praca obejmuje zarówno część teoretyczną - przedstawiającą podstawy działania kodów resztowych i ich właściwości w kontekście niezawodności układów cyfrowych - jak i część eksperymentalną, obejmującą symulację działania układu oraz analizę uzyskanych wyników.

\section{Tło i znaczenie problemu}
Wraz z miniaturyzacją technologii półprzewodnikowych rośnie podatność układów cyfrowych na zakłócenia promieniowania kosmicznego i jonizującego. Cząstki wysokoenergetyczne mogą powodować chwilowe zmiany stanów logicznych w tranzystorach, prowadząc do wystąpienia błędów przemijających (\textit{soft errors}). Tego typu zjawiska nie powodują trwałego uszkodzenia sprzętu, lecz mogą skutkować błędnymi wynikami obliczeń.

W celu ograniczenia wpływu takich błędów opracowano szereg metod detekcji i korekcji, w tym techniki oparte na kodach resztowych, które pozwalają na bieżące sprawdzanie poprawności obliczeń bez znaczącego wzrostu złożoności układu. Kody resztowe, a w szczególności kod modulo 3, umożliwiają realizację tzw. układów samosprawdzających, w których każda operacja arytmetyczna ma równoległy tor kontrolny pracujący w arytmetyce resztowej.

\newpage
\section{Założenia eksperymentu}
W ramach eksperymentu zbudowano model kombinacyjnego układu arytmetycznego składającego się z:
\begin{itemize}
    \item czterobitowego sumatora głównego (RCA i CLA),
    \item generatora reszty modulo~3 z pięciobitowego wyniku sumowania,
    \item sumatora kontrolnego realizującego operacje w arytmetyce modulo~3,
    \item komparatora porównującego reszty toru głównego i kontrolnego.
\end{itemize}

Do modelu wprowadzono mechanizm wstrzykiwania pojedynczych uszkodzeń logicznych typu \emph{stuck-at}, umożliwiający wymuszenie stałej wartości logicznej na wyjściu wybranego elementu układu. Każde uszkodzenie aktywowane jest indywidualnie i analizowane niezależnie.

Dla każdej kombinacji wejść oraz aktywnego uszkodzenia porównywany jest wynik toru głównego z wartością referencyjną, a sygnał błędu generowany przez tor kontrolny służy do klasyfikacji zachowania układu.

Wyniki symulacji klasyfikowane są jako:
\begin{itemize}
    \item \textbf{błąd wykryty (true positive)} -- wynik arytmetyczny jest niepoprawny, a układ zgłasza błąd,
    \item \textbf{błąd niewykryty (false negative)} -- wynik arytmetyczny jest niepoprawny, lecz błąd nie został zgłoszony,
    \item \textbf{fałszywy alarm (false positive)} -- wynik arytmetyczny jest poprawny, lecz układ błędnie zgłasza błąd,
    \item \textbf{brak błędu (true negative)} -- wynik poprawny i brak sygnalizacji błędu.
\end{itemize}

Takie podejście pozwala na ilościową ocenę skuteczności detekcji błędów przez układ samosprawdzający oraz identyfikację elementów najbardziej wrażliwych na uszkodzenia logiczne.
