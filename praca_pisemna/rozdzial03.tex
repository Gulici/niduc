\chapter{Plan eksperymentu/Symulacja}

\section{Struktura układu}

\begin{figure}[h]
    \centering
    \includegraphics[width=0.75\textwidth]{img04/model-układu.png}
    \caption{Schematyczny model arytmetyczego układu samosprawdzającego modulo 3. Źródło: Design of the coarse-grained reconfigurable architecture DART with
on-line error detection.\cite{JAFRI2014124}}
    \label{fig:model}
\end{figure}

\subsection{Schematy}

\begin{figure}[h]
    \centering
    \includegraphics[width=0.75\textwidth]{img04/modulo3_6bit.png}
    \caption{Schemat 6-bit generatora reszt modulo 3.}
    \label{fig:6bitmod3}
\end{figure}

\begin{figure}[h]
    \centering
    \includegraphics[width=0.75\textwidth]{img04/schemat_impl.png}
    \caption{Schemat samosprawdzającego układu arytmetyczego mod3}
    \label{fig:układ}
\end{figure}

\newpage
\section{Model uszkodzeń i symulacja}

Symulacja została zaprojektowana tak, aby odwzorować pojedyncze uszkodzenia przemijające (ang. \textit{transient faults}), których źródłem mogą być cząstki promieniowania kosmicznego powodujące chwilową zmianę stanu logicznego.  
Każde takie uszkodzenie modelowane jest w postaci sklejenia stanu logicznego \textbf{stuck-at}, polegającego na wymuszeniu stałej wartości logicznej 0 lub 1 na wyjściu danego układu logicznego.

W modelu symulacyjnym każdy układ został wyposażony w dodatkowy moduł \texttt{fault\_mux}, który umożliwia wymuszenia uszkodzenia przy pomocy magistrali sterującej \texttt{fault\_en\_bus}.  
Dzięki temu możliwe jest systematyczne testowanie wszystkich pojedynczych punktów potencjalnej awarii.

\section{Procedura testowa}

Symulacja obejmuje pełne przeszukanie przestrzeni wejść i uszkodzeń:
\[
\textit{Wejścia} \times \textit{Punkty iniekcji} \times \{\textit{stuck-at-0, stuck-at-1}\}.
\]
Dla każdej kombinacji przeprowadzane są dwa przebiegi:
\begin{enumerate}
    \item \textbf{Referencyjny} - bez aktywnego uszkodzenia, w celu uzyskania poprawnego wyniku.
    \item \textbf{Testowy} - z aktywnym pojedynczym uszkodzeniem (\texttt{fault\_en}=1), w celu porównania wyników toru głównego i kontrolnego.
\end{enumerate}

Na podstawie porównania wyników klasyfikuje się przypadki jako:
\begin{itemize}
    \item \textbf{Błąd wykryty} - wynik błędny poprawnie zgłoszony przez komparator,
    \item \textbf{Błąd niewykryty (false negative)} - wynik błędny uznany za poprawny,
    \item \textbf{Błąd fałszywy (false positive)} - wynik poprawny błędnie zidentyfikowany jako błędny.
\end{itemize}
